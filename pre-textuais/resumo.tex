% -------------------------------------------------------
% RESUMO
% -------------------------------------------------------
\begin{abstract}

A detecção de anomalias em redes de computadores é um dos pilares fundamentais para a segurança da informação, pois permite identificar tráfego suspeito, ataques cibernéticos e falhas operacionais. Entretanto, ambientes distribuídos — como organizações com múltiplas filiais, redes corporativas segmentadas ou sistemas com restrições de privacidade — apresentam desafios importantes para a centralização de dados de tráfego. Nesse contexto, técnicas tradicionais de aprendizado de máquina tornam-se limitadas devido à necessidade de reunir grandes volumes de dados sensíveis em um único local.

Este trabalho propõe uma abordagem baseada em \textit{Federated Learning} para permitir o treinamento colaborativo de modelos de detecção de anomalias sem a necessidade de compartilhamento direto de dados entre os participantes. Para aprender o comportamento normal do tráfego de rede, utiliza-se um modelo de Inteligência Artificial Generativa, como \textit{Variational Autoencoder} (VAE) ou \textit{Generative Adversarial Network} (GAN). O modelo é treinado de forma federada e desenvolve a capacidade de identificar desvios significativos através do erro de reconstrução.

O objetivo central é comparar o desempenho do modelo generativo federado com versões centralizadas e com métodos tradicionais de detecção de anomalias. Os resultados esperados incluem maior preservação de privacidade, redução no fluxo de dados entre os nós e desempenho competitivo nas métricas de classificação.

\end{abstract}