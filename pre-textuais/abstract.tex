% -------------------------------------------------------
% ABSTRACT
% -------------------------------------------------------
\renewcommand{\abstractname}{Abstract} % Garante que o título seja Abstract
\begin{abstract}

Anomaly detection in computer networks is one of the fundamental pillars of information security, as it allows for the identification of suspicious traffic, cyberattacks, and operational failures. However, distributed environments—such as organizations with multiple branches, segmented corporate networks, or systems with privacy restrictions—present significant challenges for traffic data centralization. In this context, traditional machine learning techniques become limited due to the need to gather large volumes of sensitive data in a single location.

This work proposes an approach based on \textit{Federated Learning} to enable the collaborative training of anomaly detection models without the need for direct data sharing among participants. To learn the normal behavior of network traffic, a Generative Artificial Intelligence model is used, such as a \textit{Variational Autoencoder} (VAE) or a \textit{Generative Adversarial Network} (GAN). The model is trained in a federated manner and develops the ability to identify significant deviations through the reconstruction error.



The primary objective is to compare the performance of the federated generative model with centralized versions and traditional anomaly detection methods. Expected results include enhanced privacy preservation, reduced data flow between nodes, and competitive performance in classification metrics.

\vspace{\baselineskip}
\noindent \textbf{Keywords:} Anomaly Detection. Federated Learning. Generative Artificial Intelligence. Network Security.
\end{abstract}