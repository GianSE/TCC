\chapter{Conclusão e Trabalhos Futuros}

Este trabalho apresentou uma abordagem baseada em \textit{Inteligência Artificial Generativa} combinada com \textit{Federated Learning} para detecção de anomalias em tráfego de rede. A arquitetura proposta aproveita a capacidade dos modelos generativos em aprender padrões normais e a eficiência do aprendizado distribuído para treinar modelos colaborativamente sem violar a privacidade dos dados.

\section{Conclusões}
A partir dos experimentos realizados, conclui-se que:

\begin{itemize}
    \item o modelo generativo é eficaz na detecção de anomalias por meio de erro de reconstrução;
    \item o treinamento federado apresentou desempenho próximo ao modelo centralizado;
    \item a abordagem proposta preserva privacidade e reduz custo de transmissão;
    \item o sistema é escalável e adaptável a diferentes ambientes de rede;
    \item a integração entre FL e IA generativa é viável e promissora para uso real.
\end{itemize}

Assim, o objetivo geral do trabalho foi alcançado, demonstrando que é possível combinar FL e modelos generativos para melhorar a segurança de redes distribuídas.

\section{Limitações do Trabalho}
Algumas limitações observadas incluem:

\begin{itemize}
    \item necessidade de recursos computacionais nos clientes para treinamento local;
    \item variação de desempenho conforme número de clientes e heterogeneidade dos dados;
    \item maior tempo de treinamento devido à comunicação federada.
\end{itemize}

\section{Trabalhos Futuros}
Para continuidade deste estudo, sugerem-se:

\begin{itemize}
    \item uso de \textbf{GANs federadas} e \textbf{modelos de difusão} para comparação;
    \item implementação de técnicas de \textbf{diferential privacy} ou \textbf{secure aggregation};
    \item aplicação em ambientes reais, como redes corporativas e IoT;
    \item otimização de comunicação entre clientes usando compressão de gradientes;
    \item integração com orquestradores como \textit{Kubernetes} para escalabilidade real.
\end{itemize}

Acredita-se que esses avanços podem melhorar ainda mais a eficiência e aplicabilidade da detecção distribuída de anomalias usando IA.