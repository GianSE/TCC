\chapter{Metodologia}

Este capítulo descreve a abordagem metodológica adotada para o desenvolvimento deste trabalho, incluindo arquitetura proposta, ferramentas utilizadas, preparação dos dados, definição dos modelos generativos, configuração do ambiente federado e métricas de avaliação. O objetivo é apresentar de forma clara e sistemática os passos necessários para a implementação do sistema de detecção de anomalias baseado em \textit{Federated Learning} e IA generativa.

\section{Visão Geral da Arquitetura}
A arquitetura proposta integra um modelo generativo distribuído por meio de \textit{Federated Learning}. A Figura~\ref{fig:arquitetura} apresenta uma visão geral do fluxo.

\begin{itemize}
    \item Cada cliente (nó da rede) treina localmente um modelo generativo com seus dados de tráfego.
    \item O servidor federado coleta os pesos atualizados.
    \item O agregador aplica FedAvg para gerar o modelo global.
    \item O modelo global é redistribuído aos clientes para nova rodada.
    \item A detecção de anomalias é realizada localmente usando erro de reconstrução.
\end{itemize}

\begin{figure}[H]
    \centering
    \includegraphics[width=0.95\textwidth]{figuras/arquitetura.png}
    \caption{Arquitetura geral do sistema federado para detecção de anomalias.}
    \label{fig:arquitetura}
\end{figure}

\section{Ferramentas e Tecnologias}
Para a implementação do sistema foram utilizadas as seguintes ferramentas:

\subsection{Frameworks}
\begin{itemize}
    \item \textbf{PyTorch}: para implementação dos modelos generativos.
    \item \textbf{Flower} ou \textbf{FedML}: para orquestração federada.
    \item \textbf{Scikit-learn}: pré-processamento e métricas.
    \item \textbf{Pandas / NumPy}: manipulação dos dados.
\end{itemize}

\subsection{Ambiente}
\begin{itemize}
    \item \textbf{Python 3.11}
    \item \textbf{Docker}: para simular clientes federados.
    \item \textbf{VSCode / Jupyter}: desenvolvimento do código.
\end{itemize}

\section{Descrição dos Dados}
Para avaliar o sistema, utilizou-se um dos seguintes datasets de tráfego de rede:
\begin{itemize}
    \item \textbf{CICIDS2017}
    \item \textbf{UNSW-NB15}
    \item \textbf{UGR16}
\end{itemize}

O dataset é dividido em:
\begin{itemize}
    \item dados de tráfego normal (para treinar o modelo generativo);
    \item dados contendo ataques (para avaliação da detecção).
\end{itemize}

\subsection{Pré-processamento}
As etapas incluem:
\begin{itemize}
    \item limpeza de valores ausentes;
    \item normalização Min-Max;
    \item seleção das principais características numéricas;
    \item divisão dos dados entre clientes simulados.
\end{itemize}

\section{Modelo Generativo}
O modelo generativo adotado foi um \textbf{Autoencoder Variacional (VAE)} devido à sua boa capacidade de reconstrução e estabilidade no treinamento federado.

\subsection{Arquitetura do VAE}
\begin{itemize}
    \item camadas densas no encoder e decoder;
    \item função de ativação ReLU;
    \item espaço latente de dimensão reduzida ($z$);
    \item perda composta por:
    \[
    \mathcal{L} = \text{MSE}(x, \hat{x}) + D_{KL}
    \]
\end{itemize}

Alternativamente, testes podem ser conduzidos com:
\begin{itemize}
    \item \textbf{GANs}
    \item \textbf{Diffusion Models}
\end{itemize}

\section{Configuração do Federated Learning}
O treinamento federado segue ciclos (rodadas) organizados da seguinte forma:

\begin{enumerate}
    \item Inicialização do modelo global.
    \item Distribuição para todos os clientes.
    \item Cada cliente executa $E$ épocas de treinamento local.
    \item Envio dos pesos locais ao servidor.
    \item Agregação com FedAvg.
    \item Atualização do modelo global.
\end{enumerate}

\subsection{Parâmetros utilizados}
\begin{itemize}
    \item número de clientes simulados: 3 a 10;
    \item número de rodadas federadas: 20 a 50;
    \item tamanho do batch: 64;
    \item taxa de aprendizado: 0.001.
\end{itemize}

\section{Detecção de Anomalias}
A detecção é realizada localmente, utilizando o erro de reconstrução do VAE. Para uma entrada $x$:

\[
\text{Anomalia}(x) = ||x - \hat{x}||
\]

\subsection{Threshold}
O limiar é definido com base em:
\begin{itemize}
    \item média + 3 desvios padrão dos erros de reconstrução;
    \item ou curva ROC para definição de ponto ótimo.
\end{itemize}

\section{Métricas de Avaliação}
Para avaliar o desempenho do sistema foram utilizadas:

\begin{itemize}
    \item \textbf{AUC-ROC};
    \item \textbf{Precisão};
    \item \textbf{Recall};
    \item \textbf{F1-Score};
    \item \textbf{Erro de reconstrução médio}.
\end{itemize}

Além disso, compara-se:
\begin{itemize}
    \item modelo treinado centralmente (baseline);
    \item modelo federado;
    \item impacto de diferentes números de clientes.
\end{itemize}

\section{Validação Experimental}
Os experimentos consistem em:

\begin{enumerate}
    \item Treinar o modelo centralizado.
    \item Treinar o modelo com FL.
    \item Comparar métricas.
    \item Medir impacto da heterogeneidade dos dados.
    \item Avaliar sensibilidade ao ruído.
\end{enumerate}

Esses passos permitem verificar se o aprendizado federado com IA generativa é eficaz na detecção de anomalias em cenários realistas de rede.